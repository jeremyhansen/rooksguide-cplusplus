% This work by Jeremy A. Hansen is licensed under a Creative Commons 
% Attribution-NonCommercial-ShareAlike 3.0 Unported License, 
% as described at http://creativecommons.org/licenses/by-nc-sa/3.0/legalcode


Variables are extremely important to every programmer - they will be a critical part of your programming toolkit regardless of the language you use. 
Very simply put, a variable is a space in memory that can store some range of values. 
Some of the basic data types are:

\begin{table}[tb]
	\centering
		\begin{tabular}{| l | p{3in} |}
		\hline
			\Code{int} & Short for integer; stores whole numbers \\ \hline
			\Code{char} & Short for character; stores a single letter, digit, or symbol \\ \hline
			\Code{bool} & Short for Boolean; stores \Code{true} or \Code{false} \\ \hline
			\Code{float} & Short for floating point number; stores numbers with fractional parts \\ \hline
			\Code{double} & Short for double precision floating point number; stores bigger numbers with bigger fractional parts than \Code{float} \\ \hline
		\end{tabular}
\end{table}

For a deeper discussion of data types, refer to Chapter \ref{chap_datatypes}.

\LevelD{How do I decide which data type I need?}

What you can do with a variable depends on the type of data they contain.
For instance, you can't store the number $100000$ in a \Code{char} because a \Code{char} stores only character data.
To store $100000$ the programmer should use an $int$. 
If you think you are dealing with numbers that have fractional parts, you need at least a \Code{float}. 
You generally want to use the smallest variable type that will get your job done. 
Simply put, if it is a round number, \Code{int} works fine; if it's a \Code{true} or \Code{false}, use \Code{bool}; for a letter, use \Code{char}; for fractional numbers, use \Code{float}; for a really big number or a number with many digits after the decimal point, use \Code{double}.

\LevelD{Identifiers}

Now we have an idea of what types of variables we will use in the program. 
How do we have the program differentiate between multiple \Code{int}s, \Code{char}s, or \Code{double}s? 
We have to name them! 
The name we use will give the variable an identity, so it's known as an \Keyword{identifier}. 
An identifier can be almost anything you'd like.\footnote{There are a few exceptions, including those words that describe data types (as in the table above) and other keywords such as \Code{if} and \Code{while}, which you'll learn about in later chapters.} 
Remember that the variable name can only be one word long. 
You may use a an underscore to replace a space if you so desire, and note that C++ is case sensitive. 
That is, \Code{testresults}, \Code{TestResults}, and \Code{Test\_Results} are all different identifiers.

\LevelD{Declaring a Variable}

The line of code that creates a variable is called a \Keyword{declaration}. 
A declaration is the program telling the computer ``save a place in memory for me with this name.'' 
%Maybe irrelevant/too advanced for this chapter
%Behind the scenes, the compiler gives the variable an address, which is like a map saying ``this is where your memory is.''

A declaration for an integer variable named \Code{myVariable} looks like this:

\noindent\begin{minipage}{\linewidth}\begin{lstlisting}
int myVariable;
\end{lstlisting}\end{minipage}

The specific \Keyword{syntax}---the set of grammatical rules for the language---is important to follow when declaring variables.
Notice that the first part (\Code{int}) is the data type of the variable.
The second part is the identifier (\Code{myVariable}), or variable name. 
The last part is the semicolon (\Code{;}) which signifies the end of a line. 
You can think of the semicolon in C++ as equivalent to a period at the end of a sentence; it is the end of a complete thought. 
Note that you may declare several variables of the same data type together.  Consider this example:

\noindent\begin{minipage}{\linewidth}\begin{lstlisting}
int x, y, z;
double a;
\end{lstlisting}\end{minipage}

The above example creates three variables of type \Code{int} named \Code{x}, \Code{y}, and \Code{z} and one variable of type \Code{double} named \Code{a}.  


\LevelD{Initializing Variables}

Values can be immediately assigned to a variable at the time of its declaration.  
This is known as \Keyword{initializing} a variable.  
To do this, the variable's name is followed by an equals sign (\Code{=}, the \Keyword{assignment operator}), the value, and a semicolon.  Consider this example:

\noindent\begin{minipage}{\linewidth}\begin{lstlisting}
int x = 20;
double a = 2.2;
\end{lstlisting}\end{minipage}
 
Note that uninitialized variables can cause problems if they are used anywhere before they are assigned a value. 
When a variable is declared, it contains whatever was already in that space of memory, which can give them unpredictable values. 
This means that is is often a good idea to initialize variables to some sensible initial value when they are declared.
  
\LevelD{Assignment Statements}

An assignment statement is a method of assigning a value to a variable after it has been declared.
All assignment statements have the variable being assigned the value on the left side of an equals sign and the value to assign on the right side.
Note that the expression on the right side of the assignment may contain arithmetic operations such as multiplication, division, addition, and subtraction, or even other variables.
Consider the following example:

\noindent\begin{minipage}{\linewidth}\begin{lstlisting}
int a = 1, b = 2, x = 0, y = 0;
x = a + b;
y = x;
\end{lstlisting}\end{minipage}

% TODO: character values


\LevelD{Review Questions}

\begin{enumerate}
	\item Declare two variables of type \Code{int} and initialize them to an appropriate value.
	\item Declare three integer variables: \Code{sum}, \Code{a}, \Code{b}. Initialize the variables \Code{a} and \Code{b} to an appropriate integer and use an assignment statement to assign \Code{sum} the result of \Code{a} plus \Code{b}.
	\item Declare a \Code{double} variable called \Code{number} and initialize it to $13.6$.
	%\item Declare a variable of type \Code{char}, then give the char a character.

\item Create a program in which 3 variables are declared.
Create one \Code{float} named \Code{myFloat}, one \Code{int} named \Code{myInt}, and one \Code{double} named \Code{myDouble}.
Initialize them to $3.14$, $3$, and $3.14159$, respectively. 
%Then declare a variable of type \Code{char} named \Code{myChar}, and initialize it to the character B. 

\end{enumerate}

\LevelD{Review Answers}


\begin{enumerate}
	\item \Code{int a = 6;}
	
				\Code{int b = 0;}

	\item	\Code{int sum, a = 6, b = 0;}
	
				\Code{sum = a + b;}
 
  \item \Code{double number = 13.6;}
  \item
  
\noindent\begin{minipage}{\linewidth}\begin{lstlisting}
int main()
{
  float myFloat = 3.14;
  int myInt = 3;
  double myDouble = 3.14159;

  return 0;
}
\end{lstlisting}\end{minipage}

\end{enumerate}

\LevelD{Further Reading}

\begin{itemize}
\item \url{http://www.cplusplus.com/doc/tutorial/variables/}
\item \url{http://www.tutorialspoint.com/cplusplus/cpp_variable_types.htm}
\end{itemize}	