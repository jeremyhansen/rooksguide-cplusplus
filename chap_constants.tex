We call a variable whose value we cannot change a constant. 
After you declare a constant, you are unable to change it, no matter what. 
There are two types of constants: literal and declared constants (\Code{const}). 

\LevelD{Literals}

A literal is a value outside of a variable such as $5$, $9$, $103$, and $-21$. Each of those is an \Code{int}, but a literal constant can be of any data type. The point is, these are values that the C++ compiler already recognizes, and can’t be changed. In other words, you can’t convince the compiler to give the literal 3 the value of 4, because 3 is constant. The table below contains a few examples.

13.8903
94.2321
-389283220.342423
float
‘x’
‘R’
‘%’
char
Be aware, that C++ interprets the difference between a char and a single-character variable name by the enclosure of (‘) single quotation marks.
true
false
bool
Notice that a bool only has two literal values, true or false.


Declared Constant

So what’s the difference between declaring a normal variable and a constant? When we declare a constant, we simply place the keyword const before the data type in the declaration. This indicates that whatever declaration follows the const will be a constant and cannot be changed. Since it’s a constant, you will also need to initialize the value at the same time you declare the variable. Here is an example:

const float pi = 3.14;
float radius = 5, area;

area = radius * radius * pi;
cout << area;

//program outputs 78.5 to the screen
