Pointers do just what they sound like they do. 
They point to a space in memory, usually a location occupied by a variable. 
A pointer is an address in memory. 
The pointer itself is a variable, but it also refers to a variable. 
It is declared using an asterisk following the data type:

\begin{lstlisting}
	int *ptr; 
\end{lstlisting}

The variable named \Code{ptr} is of type \Code{int*}, an ``integer pointer'' that stores the address of a variable of type \Code{int}.

To indicate that a pointer variable is not pointing toward any usable data, we often set its value to \Code{NULL}, which is defined as zero when you \Code{\#include <cstdlib>}:

\begin{lstlisting}
  int *ptr = NULL;
\end{lstlisting}

C++11 provides a dedicated null pointer object called \Code{nullptr} that can be used similarly:

\begin{lstlisting}
  int *ptr = nullptr;
\end{lstlisting}

There are two operators used in conjunction with pointers. 
The \Code{*} operator, beyond being used for multiplication and for pointer declarations, also acts as the \Keyword{dereference operator}. 
The dereference operator changes the pointer into the value it is pointing to.
It ``follows'' the address stored in the pointer and returns whatever is in that location.

The \Code{\&} operator is the \Keyword{reference operator}. 
The dereference operator returns the memory address of the variable it precedes. 
You will use this to produce a pointer to the indicated variable.

Let's declare pointer \Code{p} and use it:

\begin{lstlisting}
	int *p; // Declare an int pointer
	int var1 = 2;// Declare an int, 
	             // set it to 2
	p = &var1;   // Take the address of
	             // var1 and store it in p
	cout << *p;  // Go to the address 
	             // stored in p; return              
	             // the value; print it out
\end{lstlisting}

The output of this code is:

\noindent \Code{2}

Here is a slightly longer example:

\begin{lstlisting}
	int *p;
	int var1 = 2;
	int var2 = 4;
	p = &var1; // Take the address of
	           // var1 and store it in p
	*p = var2; // Go to the address 
	          // stored in p; assign              
	          // the value in var2
	// equivalent to var1 = var2
	cout << *p << endl;
	cout << var1 << endl;
	cout << var2 << endl;
\end{lstlisting}

The output of this code is:

\noindent \Code{4}

\noindent \Code{4}

\noindent \Code{4}

Here is the state of the variables in the second example after they are declared and initialized (lines 1-3):

% TODO: Diagram here

After the fourth lines are executed, \Code{p} will store the address of \Code{var1}:

% TODO: Diagram here

After the fifth line of code is executed, the location where \Code{p} points is assigned the value stored in \Code{var2}.
Since \Code{p} contains the address of \Code{var1}, \Code{var1} receives that value:

% TODO: Diagram here

Use caution when declaring pointers.
If you are declaring more than one pointer in a single line, make sure to indicate each pointer variable with the \Code{*} before the variable name.  
Here is a correct declaration of two pointers:

\begin{lstlisting}
	int *p, *q;
\end{lstlisting}

This results in an integer pointer named \Code{p} and an integer pointer named \Code{q}. 
Contrast that with the below code:

\begin{lstlisting}
	int *p, q;
\end{lstlisting}

This results in an integer pointer named \Code{p} and an integer named \Code{q}. 
An equivalent way to write the above is:

\begin{lstlisting}
	int q, *p;
\end{lstlisting}

\LevelD{Review Questions}

\begin{enumerate}
	\item What is the output of the following code?
  \begin{lstlisting}
	int *a, b, c;
	a = &b;
	b = 5;
	c = 1;
	b = b - b;
	c = b * b;
	*a = c - *a;
	a = &c;
	*a = c - 7;
	c = c + c;
	*a = *a + b;
	c = c + b;
	b = c - 3;
	c = *a - 7;
	cout << *a << endl;
	cout << b << endl;
	cout << c << endl;
	\end{lstlisting}
	\item What is the output of the following code?
  \begin{lstlisting}
	int a, b, *c;
	a = 7;
	b = 4;
	c = &a;
	a = *c - a;
	*c = *c + 4;
	a = b + a;
	c = &b;
	a = a - b;
	*c = b + a;
	*c = *c - 1;
	a = a * 1;
	a = b - *c;
	a = a - *c;
	cout << a << endl;
	cout << b << endl;
	cout << *c << endl;
	\end{lstlisting}
\end{enumerate}

\LevelD{Homework Questions}

\LevelD{Review Answers}

\begin{enumerate}
	\item \Code{-21}

				\Code{-17}

				\Code{-21}

	\item \Code{-7}

				\Code{7}

				\Code{7}
\end{enumerate}

\LevelD{Homework Answers}

\LevelD{Further Reading}

\begin{itemize}
\item ~
\item ~
\item ~
\end{itemize}	
