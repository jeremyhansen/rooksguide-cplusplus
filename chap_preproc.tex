% This work by Jeremy A. Hansen is licensed under a Creative Commons 
% Attribution-NonCommercial-ShareAlike 3.0 Unported License, 
% as described at http://creativecommons.org/licenses/by-nc-sa/3.0/legalcode

Preprocessor directives are lines of code that are resolved before the compilation of the code begins.
Directives can be used to change source code prior to compilation.
This may include adding, removing, or changing values in the source of the program.
One of the most frequently-used preprocessor directives is \Code{\#include}.

\LevelD{\Code{\#include}}

When we want to include a separate file for the compilation process, we use the directive \Code{\#include} followed by the library or file name. 
The distinguishing factor between including libraries and including files is the use of angle brackets and quotes, respectively. 
For example, when we want to include the \Code{iostream} library: 

\noindent\begin{minipage}{\linewidth}\begin{lstlisting}
#include <iostream> 
\end{lstlisting}\end{minipage}

If we want to include a file, such as a file named \Code{myFile.h}, we can write: 

\noindent\begin{minipage}{\linewidth}\begin{lstlisting}
#include "myFile.h"
\end{lstlisting}\end{minipage}

However, when we include files, they must be in the same directory as the file where the \Code{\#include} appears. 
We discuss the C++ Standard Library in Chapter \ref{chap_stl}, and include a short sample of other libraries in Table \ref{tab:libraries}.

\begin{table}[tb]
	\centering
		\begin{tabular}{| l | p{1in} | p{1.8in} |}
		\hline
			\textbf{Library} & \textbf{Provides} & \textbf{Some common uses} \\ \hline
			
			\Code{<iostream>} & Input/output stream objects & \Code{cout}, \Code{cin}: \newline \mbox{see Chapters~\ref{chap_input}~and~\ref{chap_output}} \\ \hline
			\Code{<cstdlib>} & The C standard library & \Code{rand()}, \Code{abs()}, \Code{NULL} \\ \hline
			\Code{<cmath>} & Mathematical functions & \Code{pow()}, \Code{sqrt()}, \newline \Code{cos()}, \Code{tan()}, \Code{sin()}: \newline \mbox{see Chapter~\ref{chap_advancedarith}} \\ \hline
			\Code{<iomanip>} & Input/output manipulation & \Code{set\_iosflags()}, \newline \Code{setfill()}, \newline \Code{setprecision()} \\ \hline
			\Code{<ctime>} & Time-related functions & \Code{clock()}, \Code{time()} \\ \hline
			\Code{<string>} & The \Code{string} class & See Chapter~\ref{chap_strings} \\ \hline
			\Code{<fstream>} & File input and output streams & See Chapter~\ref{chap_file_io} \\ \hline
				
		\end{tabular}
		\caption{Some useful libraries and a sampling of what they provide} \label{tab:libraries}
\end{table}

%TODO: #if, #ifdef, #ifndef, #define, etc.
\LevelD{\Code{\#define}}

Syntax:

\Code{\#define \textit{identifier} \textit{token-string}}

\Code{\#define \textit{identifier(args...)} \textit{token-string}}
~\\

The \Code{\#define} directive defines an \textit{identifier} which is replaced by a \textit{token-string} in the source file.
The preprocessor will replace all instances of the \textit{identifier} with instances of the \textit{token-string} prior to compilation.

The \textit{token-string} may be any combination of keywords, constants, or complete statements. For example, to define a macro to represent a constant:
~\\

\Code{\#define PI 3.14159}
~\\

A macro may also define a statement:

~\\

\Code{\#define ADD(a, b) (a + b)}

~\\

These could be used in the source code like this:

\noindent\begin{minipage}{\linewidth}\begin{lstlisting}

#include <iostream>

#define PI 3.14159
#define ADD(a, b) (a + b)

using namespace std;

int main()
{
  cout << ADD(PI, PI) << endl;
}

\end{lstlisting}\end{minipage}

After the preprocessor processes the file prior to compilation the macros are expanded to this:


\noindent\begin{minipage}{\linewidth}\begin{lstlisting}

#include <iostream>

#define PI 3.14159
#define ADD(a, b) (a + b)

using namespace std;

int main()
{
  cout << (3.14159 + 3.14159) << endl;
}

\end{lstlisting}\end{minipage}

\LevelD{\Code{\#undef}}

Syntax:

\Code{\#undef \textit{identifier}}

~\\

The \Code{\#undef} directive undefines a previously defined \textit{identifier}.

\LevelD{\Code{\#if}, \Code{\#elif}, \Code{\#else}, \Code{\#endif}}

Syntax:

\Code{\#if \textit{conditional}}

\Code{\#elif \textit{conditional}}

\Code{\#else}

~\\

If the conditional is non-zero the lines following the \Code{\#if} are parsed until an \Code{\#endif} is reached. \Code{\#elif} is the equivalent of the compiled \textit{else if}. If the \textit{conditional} is non-zero, the lines following are parsed until the next directive is reached.

\LevelD{\Code{\#ifdef}, \Code{\#ifndef}}

Syntax:

\Code{\#ifdef \textit{identifier}}

\Code{\#elif \textit{identifier}}

~\\

The \Code{\#ifdef} and \Code{\#ifndef} directives are similar to the \Code{\#if} directive. However, \Code{\#ifdef} checks to see if an \textit{identifier} is defined and if so, the condition is true. \Code{\#ifndef} checks to see if an identifier is not defined and if so, the condition is true.

\LevelD{\Code{\#using}}

Syntax:

\Code{\#using \textit{identifier}}

~\\

The \Code{\#using} directive imports metadata from an external .dll, .exe, .netmodule, or .obj. Do not confuse this with \textit{using namespace}, as they are not the same thing. 

~\\

\newpage

\LevelD{Review Questions}

\begin{enumerate}
	\item Which of the following demonstrate correct syntax for \Code{\#include} statements? (Note: some of these may be syntactically correct but not do what you would expect!)
	
	\begin{enumerate}
		\item \Code{\#include <aFile.txt>}
		\item \Code{\#include <iostream>;}
		\item \Code{include <iostream>}
		\item \Code{\#include myFile.txt;}
		\item \Code{\#include "myFile.txt"}
		\item \Code{\#include <cmath>;}
		\item \Code{include <cmath>}
		\item \Code{include "cmath"}
		\item \Code{\#include <cmath>}
		\item \Code{\#include (iostream);}
		\item \Code{\#include <iostream>}
	\end{enumerate}
 
	\item Identify the the preprocessor statements in the following code: \nopagebreak[4]

\noindent\begin{minipage}{\linewidth}\begin{lstlisting}
#include <cstdlib>
#include <iostream>
using namespace std;
int main()
{
  cout << "Included!" << endl;
  return 0;
}
\end{lstlisting}\end{minipage}

  \item Which library is required to use the \Code{cout} object?
 
	\item Is \Code{using namespace std;} a preprocessor directive?
 
	\item If you want to be able to use the funtion \Code{pow()}, which library do you need?

\end{enumerate}

\LevelD{Review Answers}

\begin{enumerate}
	\item \textbf{a}, \textbf{e}, \textbf{i}, and \textbf{k}.
	\item The first two lines are preprocessor directives.
	\item The \Code{iostream} library.
	\item No.
	\item The \Code{cmath} library.
\end{enumerate}

%\LevelD{Further Reading}

%\begin{itemize}
%\item \url{~}
%\item \url{~}
%\item \url{~}
%\end{itemize}	

